\documentclass[10pt]{article}

\usepackage[hidelinks]{hyperref}

\setlength{\parindent}{0pt} \setlength{\oddsidemargin}{0in}
\setlength{\evensidemargin}{0in} \setlength{\textwidth}{6.5in}
\setlength{\headheight}{0in} \setlength{\headsep}{0in}
\setlength{\topmargin}{-0.2in} \setlength{\textheight}{9.3in}
\setlength{\parsep}{0pt} \setlength{\parskip}{0pt}
\setlength{\partopsep}{1pt} 
\setlength{\topsep}{2pt}
%\setlength{\itemsep}{-5pt}
%\setlength{\labelwidth}{6pt}
%\setlength{\labelsep}{0pt}

\renewcommand{\section}[1]{ \vspace{10pt}\begin{flushleft}{\hspace{-0.2in }\Large\bf
    #1}\end{flushleft}\nopagebreak }

\renewcommand{\subsection}[1]{ \vspace{2pt}\begin{flushleft}{\hspace{-0.1in} \large\it
    #1}\end{flushleft}\nopagebreak }


\renewcommand{\thebibliography}[1]{ \list
  {[\arabic{enumi}]}{\settowidth\labelwidth{[#1]}\leftmargin\labelwidth
    \advance\leftmargin\labelsep \usecounter{enumi}}
  \def\newblock{\hskip .11em plus .33em minus -.07em} \sloppy
  \sfcode`\.=1000\relax}

\begin{document}

\begin{center}
\vspace{-1.5in}
  {\Large\bf Zoran Dimitrijevi\'{c}}\\
  {\large\em Curriculum Vitae}
\end{center}

\begin{center}
{\em zorand@gmail.com}\\
+1 (805) 259-5262\\
\url{http://3opan.net/~zorand}
%http://3opan.net/$\sim$zorand
\end{center}

%\section{Vita}
%
%\begin{tabular}{ll}
%  Birthday: \hspace{0.20in} 	& November 16, 1974. \\
%  Birthplace: 			& \v{S}abac, Serbia, Yugoslavia.
%\end{tabular}

\vspace{-5pt}
\section{Experience Summary}

\begin{tabular}{ll}
2019-present    & Independent consultant.\\
2017-2019       & SAP Labs, Palo Alto, California.\\
2014-2017       & Altiscale Inc. acquired by SAP Labs, Palo Alto, California.\\
2004-2013       & Google Inc., Mountain View, California.\\
1999-2004	& Teaching and Research Assistant, 
			CS and ECE Department, UCSB. \\
Summer 2003	& Instructor (Teaching Associate), CS Department, UCSB. \\
Summer 2001	& Research Scientist, Sony 550 Digital Media Ventures, 
			San Francisco, California. \\
Summer 2000	& Research Intern, HP Labs, Palo Alto, California. \\
1998-1999	& Research Assistant, CE\&CS Department, University of Belgrade, Serbia.\\
Fall 1997	& Research Intern, ECE Department, University of Campinas, Brazil. \\
\end{tabular}

\vspace{-5pt}
\section{Technical Interests}
Design and implementation of large-scale storage systems, quality of service, 
parallel and cluster-based computing, multimedia systems, and large-scale 
search engines. Security, privacy and encryption. News and citizen journalism. 
Open Source.


\section{Education}

\begin{tabular}{ll}
June 2004 \hspace{0.16in}	& \textbf{Ph.D. in Computer Science}, 
		University of California, Santa Barbara (UCSB). \\
        & Dissertation title: ``Quality-of-Service Scheduling in Storage Systems''\\
	&  Advisor: { Prof. Edward Y. Chang}. \\

2003		& \textbf{M.S. in Computer Science}, 
		University of California, Santa Barbara.\\ & GPA: 4.00/4.00. \\
1999 		& \textbf{Dipl.Ing. in Electrical Engineering}, 
		School of Electrical Engineering,\\
 & 		University of Belgrade, Serbia, Yugoslavia. GPA: 8.95/10.00.
\end{tabular}

%\vspace{-5pt}
\section{Experience}

{\bf Principal Software Engineer,\\
SAP Labs, Palo Alto, California, 2017--2019.}
\begin{itemize}
\vspace{-5pt}
\item
Tech lead for Hadoop team (open-source components of SAP Cloud Platform Big Data Services including Apache Hadoop, 
Hive, Ranger and Tez). 
Service owner for internal infrastructure for Altiscale CA and x509 certificates, data encryption-at-rest, 
Kerberos, Hadoop build and deploy pipelines, and data transfer tools.
Tech lead for extending Apache Ranger for SAP Big Data Services (https://github.com/ altiscale/ranger). 
Launched Apache Ranger for SAP BDS in production. Launched SSL support for Altiscale HttpFs (and fixed a bug in open-source for swebhdfs tokens). 
Prototyped Altiscale Hadoop and Spark running on Kubernetes. Prototyped autoscaling Spark when running on 
Apache Yarn (on Kubernetes and on AWS). Implemented templates for microservices in Go with client-cert authentication and
ported Altiscale encryption-at-rest key management service in Go. 
\end{itemize}


{\bf Principal Software Engineer,\\
Altiscale Inc., Palo Alto, California, 2014--2017 (acquired by SAP in September 2016).}
\begin{itemize}
\vspace{-5pt}
\item
Tech lead for Data transfer tools. Improved open-source Hadoop distcp by multithreading file listings which were 
slow for large number of S3 objects (in some cases improvement from 60 minutes to 2 minutes). Implemented java-based
tool for rsync between HDFS and S3 (and Glacier). Refactored Altiscale ruby-based client copytohdfs for high-volume 
enterprise customers.
Wrote chef-server recipes for Altiscale disk partitioning and mkfs supporting dockerized environments.
Designed and implemented Altiscale encryption-at-rest for HDFS based on 
Linux dmcrypt+LUKS block-device encryption and client-cert based secure-hash infrastructure. 
\item
Tech lead for Hadoop team. 
Owned and maintained several production chef-server recipes and Jenkins pipelines 
including core Hadoop and internal x509 CA services. Various performance evaluation studies, especially for containerized environments and encrypted disks and SSDs.
Designed and implemented open-source tool for multiplexing HttpFs transfers via multiple ssh tunnels: https://github.com/ altiscale/transfer-accelerator. Helped certify Altiscale as SAP acquired-to-ship product, especially for data-transfer tools and data encryption-at-rest. Helped with compliance certifications.
%\item
%September 1, 2016: Altiscale acquired by SAP.
\end{itemize}

%\pagebreak

{\bf Software Engineer and Senior Software Engineer,\\
Google Inc., Mountain View, California, 2004--2013.}
\begin{itemize}
\vspace{-5pt}
\item
2004--2005: Systems Infrastructure, Google File System (GFS) team. Worked with Howard Gobioff as my TL in Bill Coughran's team. Added background checksumming in GFS Chunkserver (C++) which run on most Google machines in production until migration from GFS to Collosus. Worked on improving integration testing infrastructure for GFS (injecting faults, Python).
As a 0\% project launched www.google.co.yu. Launched www.google.cs. Launched www.google.rs after adding support to Google Geo-IP for my former country which kept changing names and borders. As a 20\% project, designed and prototyped Google Video Streaming Service (buffering layer for videos stored on GFS) which was later used as a starting architecture for YouTube streaming infrastructure.
\item
2005--2008: Systems Infrastructure, Google News Archive Search. As part of a small team with Yanghua Chu and Anurag Acharya (and later Dale Neal) designed and implemented Google News Archive Search. Implemented Google premium crawler. Implemented C++ frontend for News Archive Search (arfe). We launched News Archives in 2006 as part of news.google.com. Wrote all arfe-related borgcfg and related GFE and rpc-balancer rules. Worked with SRE teams to get required production Borg resources. Worked with PM teams to get all UI approved. Implemented most UI and Google-data-push (GDP) infrastructure for News Archives. Implemented all borgmon and related varz interfaces for arfe. Implemented backend APIs for Google one-box and news.google.com in arfe. Handled most of google.com traffic as Google News backend. Added support for pre-1970 dates for Google Search. Implemented and launched News Archive Search in several European languages. Launched Google Timeline. Implemented iGoogle module based on Archive Search. Worked with a team in India to include OCR documents from microfilms to News Archive Search.
\item
2009--2011: Google News. Technical owner for videos in Google News. Added user generated content from YouTube into Google News index. Designed, implemented and lunched this feature in production. Added news-video pubsub stream for Fresh Docs (Google real-time index at that time). Re-implemented search result pages for Google News back in GWS as part of One Google project (Google-wide project to provide unified infrastructure for search-result page rendering). On-call in production for Google News and News Archives. Launched Google News in several new languages including Serbian news.google.rs.
\item
2012: YouTube. Joined YouTube to start a new experimental project YouTube News.
%\item
%2013: Google Local Search. Tech Lead for Local listings.
\end{itemize}

{\bf Graduate Research Assistant for Prof. Edward Y. Chang,\\
CS and ECE Department, UCSB, Santa Barbara, California, 2001--2004.}
\begin{itemize}
\vspace{-5pt}
\item 
{Proposed preemptible disk scheduling algorithms. Investigated  
 preemptibility of disk IOs. Designed and implemented {\em Semi-preemptible IO} 
 prototype. Proposed and designed preemptive RAID scheduling algorithms. 
 Designed IO preemption and resumption criteria.  
 Implemented a simulator for preemptible RAID systems ({\em PraidSim}). 
 %(Disksim-based simulator of preemptible RAID systems). 
 Research in the areas of disk profiling, modeling, and data storage.
 Designed and implemented QoS extensions for Linux disk scheduling ({\em UCSB-IO}).}

\vspace{-5pt}
\item 
 Co-designed MEMS-based disk buffer for streaming media servers. 
 Co-designed the analytical framework, admission-control criteria, and data-placement 
 algorithms.

\vspace{-5pt}
\item 
Design and implementation of SfinX video surveillance system.
Proposed the hardware and software architecture for SfinX. 
Co-designed and co-implemented Xtream video streaming and storage system.
Investigated video streaming over wireless networks.
%Co-designed and co-implemented disk admission control algorithms.
\end{itemize}


%\begin{tabular}{ll}
%{\bf Summer 2001}  & {\bf Research Intern for Dr. Sean Varah, Sony 
%550 Digital Media Ventures,}\\
% & {\bf San Francisco, California.}
%\end{tabular}


{\bf Research Scientist (Internship) for Dr. Sean Varah,\\
Sony 550 Digital Media Ventures, San Francisco, California, Summer 2001.}
\begin{itemize}
\vspace{-5pt}
\item
Research in the design and implementation of real-time storage systems for streaming media.
\end{itemize}


%\begin{tabular}{ll}
%{\bf Summer 2000}  & {\bf Research Intern for Dr. Dejan Miloji\v{c}i\'c,
%			HP Labs, Palo Alto, California.}
%\end{tabular}

%\pagebreak

{\bf Research Intern for Dr. Dejan Miloji\v{c}i\'c, HP Labs, Palo Alto, California, Summer 2000.}
\begin{itemize}
\vspace{-5pt}
\item
Researched the susceptibility of operating systems and software to soft errors.
\end{itemize}


%\pagebreak

%\begin{tabular}{ll}
%{\bf Winter 2000}  & {\bf Independent Research with Prof. Anurag Acharya, 
%			CS Department, UCSB. }
%\end{tabular}

%{\bf Independent Research with Prof. Anurag Acharya, CS Department, UCSB, Winter 2000. }
%\begin{itemize}
%\vspace{-5pt}
%\item
% Designed and implemented a disk-profiling tool ({\em Diskbench}). 
% Co-implemented open-source library for user-level access
% to SCSI devices. Investigated problems involved with large-scale cluster 
% architectures using UCSB Gargleblaster Beowulf cluster.
%\end{itemize}


%\begin{tabular}{ll}
%{\bf 1998-1999} \hspace{0.15in}   & {\bf Research Assistant for 
%			Prof. Veljko Milutinovi\'c, CE\&CS Department,} \\
% & {\bf School of Electrical Engineering, University of Belgrade, Serbia.} 
%\end{tabular}

%{\bf Research Assistant for Prof. Veljko Milutinovi\'c, CE\&CS Department, 
%School of Electrical Engineering, University of Belgrade, Serbia, June 1998--July 1999.} 
%\begin{itemize}
%\vspace{-5pt}
%\item
%Designed and implemented extensions of an execution-driven shared memory simulator 
%({\em Limes}) to enable simulation of distributed shared-memory systems. 
%Worked on the design and architecture of a system-on-a-chip GPS handheld 
%cartographic display system. Worked on the design of split-cache architectures. 
%\end{itemize}


%\begin{tabular}{ll}
%{\bf Fall 1997}  \hspace{0.20in}  & {\bf Research Intern (IAESTE exchange program) for 
%		Prof. Ariovaldo Garcia,} \\
% & 		{\bf ECE Department, University of Campinas, Brazil.}
%\end{tabular}

%{\bf Research Intern (IAESTE exchange program) for Prof. Ariovaldo Garcia, 
%ECE Department, \\
%University of Campinas, Brazil, Fall 1997.}
%\begin{itemize}
%\vspace{-5pt}
%\item
%Worked on message-passing programming in C for nCube and Parallel Virtual Machine 
%({\em PVM}) for Solaris clusters.
%\end{itemize}


\pagebreak

\section{Publications}
\begin{thebibliography}{22.}

%\subsection{Ph.D. Dissertation}
\vspace{-6pt}

\bibitem{zoran-thesis}
 Zoran Dimitrijevi\'c. 
 \newblock {\bf Quality of service scheduling in storage systems.} 
 \newblock Ph.D. Dissertation (ISBN 0-496-84077-0), Department of Computer Science, 
 University of California, Santa Barbara, June 2004.
%\end{thebibliography}


\subsection{Journal Papers}
%\begin{thebibliography}{22.}
\vspace{-6pt}

\bibitem{ts2007}
 R.~Rangaswami, Z.~Dimitrijevi\'c, E.~Chang, and K.~Schauser.
 \newblock {\bf Building MEMS-based Storage Systems for Streaming Media.}
 \newblock {\em ACM Transactions on Storage}, Volume 3 Issue 2, June 2007.

\bibitem{tc2005}
 Z.~Dimitrijevi\'c, R.~Rangaswami, and E.~Chang.
 \newblock {\bf Systems Support for Preemptive Disk Scheduling.}
 \newblock {\em IEEE Transactions on Computers}, pages 1314--1326, October 2005.

\bibitem{hpl-tc-softerrors}
 A.~Messer, P.~Bernadat, G.~Fu, D.~Chen, Z.~Dimitrijevi\'c, 
 D.~Lie, D.~Mannaru, A.~Riska, and D.~Miloji\v{c}i\'c.
 \newblock {\bf Susceptibility of commodity systems and software to 
 memory soft errors.}
 \newblock {\em IEEE Transactions on Computers}, pages 1557--1568, December 2004.


\bibitem{tmm03}
 R.~Rangaswami, Z.~Dimitrijevi\'c, E.~Chang, and S.-H.~G.~Chan.
 \newblock {\bf Fine-grained device management in an interactive media server.}
 \newblock {\em IEEE Transactions on Multimedia}, pages 558--569, December 2003.

\bibitem{tcca1}
 M.~Prvulovi\'c, D.~Marinov, Z.~Dimitrijevi\'c, and V.~Milutinovi\'c.
 \newblock {\bf Split temporal/spatial cache: A survey and reevaluation of performance.}
 \newblock {\em Newsletter of Technical Committee on Computer Architecture}, 
 pages 8--17, IEEE Computer Society, July 1999.

\bibitem{tcca2}
 M.~Prvulovi\'c, D.~Marinov, Z.~Dimitrijevi\'c, and V.~Milutinovi\'c.
 \newblock {\bf The split spatial/non-spatial cache: A performance and complexity evaluation.}
 \newblock {\em Newsletter of Technical Committee on Computer Architecture}, 
 pages 18--25, IEEE Computer Society, July 1999.


\subsection{Conference Papers}
\vspace{-6pt}

\bibitem{ic2e2019}
 Z.~Dimitrijevi\'c, C.~Sahin, C.~Tinnefeld, and J.~Patvarczki
 \newblock {\bf Importance of Application-level Resource Management in Multi-cloud Deployments (invited).}
 \newblock {\em Accepted to IEEE International Conference on Cloud Engineering (IC2E)}, June 2019.

\bibitem{rtss2005}
 B.~Liu, R.~Rangaswami, and Z.~Dimitrijevi\'c. 
 \newblock {\bf Thwarting virtual bottlenecks in multi-bitrate streaming
 servers.}
 \newblock {\em Proceedings of IEEE Real-Time Systems Symposium (RTSS)}, December 2005.

\bibitem{icme2004}
 R.~Rangaswami, Z.~Dimitrijevi\'c, K.~Kakligian, E.~Chang, and Y.-F.~Wang. 
 \newblock {\bf The SfinX video surveillance system.}
 \newblock {\em Accepted to IEEE International Conference on Multimedia and
 Expo (ICME)}, June 2004.

\bibitem{sfinx-pcm2003}
 Z.~Dimitrijevi\'c, G.~Wu, and E.~Chang.
 \newblock {\bf {SFINX}: {A} multi-sensor fusion and mining system (invited).}
 \newblock {\em Proceedings of the IEEE Pacific-rim Conference on Multimedia},
 Singapore, December 2003.

\bibitem{ipsi2003} 
 Z.~Dimitrijevi\'c and R.~Rangaswami.
 \newblock {\bf Quality of service support for real-time storage systems (invited).} 
 \newblock {\em Proceedings of International IPSI-2003 Conference}, 
 St. Stefan, Montenegro, October 2003.

\bibitem{spio-fast03}
 Z.~Dimitrijevi\'c, R.~Rangaswami, and E.~Chang.
 \newblock {\bf Design and implementation of {Semi-preemptible IO}.}
 \newblock {\em Proceeding of the Second Usenix File and Storage Technology (FAST)}, 
 pages 145--158, San Francisco, California, March 2003.

\bibitem{raju03-icde}
 R.~Rangaswami, Z.~Dimitrijevi\'c, E.~Chang, and K.~E.~Schauser. 
 \newblock{\bf MEMS-based disk buffer for streaming media servers.}
 \newblock {\em Proceedings of the 19th IEEE International Conference on Data
 Engineering (ICDE)}, pages 619--630, Bangalore, India, March 2003.

\bibitem{vio-short}
 Z.~Dimitrijevi\'c, R.~Rangaswami, and E.~Chang. 
 \newblock {\bf Virtual IO: Preemptible disk access.\footnote{Extended version
 of this paper is published as Semi-preemptible IO~\cite{spio-fast03}.}}
 \newblock {\em Proceedings of the 10th ACM Conference on Multimedia}, 
 pages 231--234, Juan Les Pins, France, December 2002.

\bibitem{xtream}
 Z.~Dimitrijevi\'c, R.~Rangaswami, and E.~Chang.
 \newblock {\bf The {XTREAM} multimedia system.}
 \newblock {\em Proceedings of IEEE Conference on Multimedia and Expo}, pages 545--548, 
 Lausanne, Switzerland, August 2002.

\bibitem{jvm}
 D.~Chen, A.~Messer, P.~Bernadat, G.~Fu, Z.~Dimitrijevi\'c, 
 D.~Lie, D.~Mannaru, A.~Riska, and D.~Miloji\v{c}i\'c.
 \newblock {\bf JVM susceptibility to memory errors.}
 \newblock {\em Proceedings of Usenix Java[tm] Virtual Machine Research and Technology 
 Symposium}, pages 67--77, Monterey, California, April 2001.
%\end{thebibliography}


\subsection{Technical Reports}
%\footnote{Some of the techical reports are
%subsequently extended and published as conference or journal papers.}
%\begin{thebibliography}{22.}
\vspace{-6pt}
\bibitem{fiu-stream}
 B.~Liu, R.~Rangaswami, and Z.~Dimitrijevi\'c.
 \newblock {\bf Thwarting virtual bottlenecks in multi-bitrate streaming
 servers.}
 \newblock FIU Technical Report TR-2005-10-02, 2005.\footnote{Subsequently
 published as a conference paper~\cite{rtss2005}.}

%\bibitem{praid-04}
% Z.~Dimitrijevi\'c, R.~Rangaswami, and E.~Chang.
% \newblock {\bf Preemptive RAID scheduling.}
% \newblock UCSB Technical Report TR-2004-19, 2004.

\bibitem{diskbench}
 Z.~Dimitrijevi\'c, R.~Rangaswami, D.~Watson, and A.~Acharya. 
 \newblock {\bf Diskbench: User-level disk feature extraction tool.}
 UCSB Technical Report TR-2004-18, 2004.

\bibitem{scsibench} 
Z.~Dimitrijevi\'c, R.~Rangaswami, D.~Watson, and A.~Acharya.
\newblock {\bf User-level SCSI disk feature extraction.}
\newblock Technical Report, UCSB, July 2001.

\bibitem{hpl-softerrors}
 A.~Messer, P.~Bernadat, G.~Fu, D.~Chen, Z.~Dimitrijevi\'c, 
 D.~Lie, D.~Mannaru, A.~Riska, and D.~Miloji\v{c}i\'c.
 \newblock {\bf Susceptibility of modern systems and software to soft errors.}
 \newblock {\em HP Labs Technical Report HPL-2001-43}, 2001.\footnote{Subsequently
 published as a journal paper~\cite{hpl-tc-softerrors}.}

\bibitem{gps1}
 I.~Ikodinovi\'c, Z.~Dimitrijevi\'c, V.~Milutinovi\'c, and A.~Prete.
 \newblock {\bf GPS-enabled handheld cartographic display systems: A survey.}
 \newblock Technical Report, School of Electrical Engineering, 
 University of Belgrade, 2000.

\bibitem{gps2} 
 I.~Ikodinovi\'c, Z.~Dimitrijevi\'c, V.~Milutinovi\'c, and A.~Prete.
 \newblock {\bf Proposing the architecture for a high-performance GPS-enabled handheld cartographic display system.}
 \newblock Technical Report, School of Electrical Engineering, 
 University of Belgrade, 2000.


\subsection{Other Publications}

\vspace{-6pt}

\bibitem{praid-04}
 Z.~Dimitrijevi\'c, R.~Rangaswami, and E.~Chang.
 \newblock {\bf Architectural Support for Preemptive RAID schedulers.}
 \newblock Usenix FAST WiP Report/Poster, March, 2004.

\bibitem{limes-book}
 I.~Ikodinovi\'c, Z.~Dimitrijevi\'c, D.~Magdi\'c, A.~Milenkovi\'c, J.~Proti\'c, and 
 V.~Milutinovi\'c.
 \newblock {\bf Limes: A multiprocessor simulation environment for PC platforms.}
 \newblock Book chapter in {\em Microprocessor and Multimicroprocessor Systems} 
 by Veljko Milutinovi\'c, ISBN 0471357286, John Wiley \& Sons, Inc., 2000.

\bibitem{diplomski}
 Z.~Dimitrijevi\'c.
 \newblock {\bf Software environment for the simulation of distributed shared memory 
 multiprocessor systems.}
 \newblock Engineering Diploma Thesis (in Serbian), School of Electrical Engineering, 
 University of Belgrade, Serbia, July 1999.
\end{thebibliography}


\vspace{-5pt}
\hspace{0.25in} (Full papers are available on-line at 
\url{http://3opan.net/~zorand/publications.html}.)


\section{Teaching Experience}

\begin{tabular}{ll}
{ Summer 2003} & 
 {\bf Instructor (Teaching Associate)}, CS Department, Summer Sessions, UCSB.\\
 & {\em CS170 Operating Systems} (upper-division undergraduate course).\\
%Organized and lectured course of 46 students with the help of two Teaching Assistants.

{ Winter 2003} & 
 {\bf Teaching Assistant for Prof. Tao Yang}, CS Department, UCSB. \\
 & {\em CS240B High Performance Computing Systems and Applications} (graduate-level course).\\

{ Winter 2001} & 
 {\bf Teaching Assistant for Prof. Peter Cappello}, CS Department, UCSB. \\
 & {\em CS172 Software Engineering} (upper-division undergraduate course).\\

{ Fall 2000} & 
 {\bf Teaching Assistant for Prof. Alan Konheim}, CS Department, UCSB. \\
 & {\em CS176A Computer Networks} (upper-division undergraduate course) \\

{ Spring 2000} & 
 {\bf Teaching Assistant for Prof. Klaus Schauser}, CS Department, UCSB. \\
 & {\em CS290I Scalable Internet Services and Systems} (graduate-level course).\\

{ Winter 2000} & 
 {\bf Teaching Assistant for Prof. Amr El Abbadi}, CS Department, UCSB. \\
 & {\em CS130A Data Structures and Algorithms I} (upper-division undergraduate course).\\

{ Fall 1999}   & 
 {\bf Teaching Assistant for Prof. Anurag Acharya}, CS Department, UCSB. \\
 & {\em CS170 Operating Systems} (upper-division undergraduate course).\\
\end{tabular}



%\pagebreak

\section{Selected Talks}
\vspace{-6pt}
\begin{thebibliography}{22.}

\bibitem{ucsb-2011} 
{\bf YouTube News: Broadcasting the World.} 
Industry Keynote, UCSB Graduate Student Workshop, October 2011.

\bibitem{ucb-talk} 
{\bf Google: organizing the world's information... and loving it.} Google
recruiting tech talk, University of Colorado at Boulder, October
2005.

\bibitem{fiu-talk} 
{\bf Google: a computer scientist's playground.} Invited talk at 
Computer Science Department, FIU, Miami, April 2005.

\bibitem{rich-class} 
{\bf Two Pieces of Google Core Infrastructure: GFS and MapReduce.} 
Invited lecture for Prof. Rich Wolski's CS290B Advanced Operating Systems 
course, UCSB, March 2005.

%\bibitem{etf-talk} 
%{\bf Google: a computer scientist's playground.} Invited talk at 
%School of Electrical Engineering, University of Belgrade, Serbia, December
%2004.

\bibitem{utk-talk} 
{\bf Google: a computer scientist's playground.} Invited talk at 
Innovative Computing Laboratory, University of Tennessee at Knoxville, October 2004.

\bibitem{ibm-talk} 
{\bf High-performance preemptible and MEMS-based IOs.} Invited talk at 
IBM Almaden \\Research Center, Almaden, California, November 2003.

%\bibitem{ipsi2003-talk} 
%{\bf Using open source in academia for computer science research and education.}
%\\Tutorial talk at IPSI-2003, St. Stefan, Montenegro, October 2003.

\bibitem{fast-talk} 
{\bf Design and implementation of Semi-preemptible IO.} Conference talk at 
Usenix FAST'03, \\San Francisco, California, April 2003.

\bibitem{gl1} 
{\bf Multiprocessor systems.} Guest lecture for 
ECE154 Computer Architecture, UCSB, Winter 2003.

\bibitem{gl2} 
{\bf I/O storage.} Guest lecture for ECE154 Computer Architecture, 
UCSB, Winter 2003.

%\bibitem{gl3} 
%{\bf QoS scheduling for storage systems.} Guest lecture for ECE160 
%Multimedia Systems, UCSB,\\ Spring 2003.

\bibitem{ucla-vio}
{\bf Virtual IO: Preemptible disk access.} Research seminar for Prof. Richard 
Muntz's group, UCLA, August 2002.

%\bibitem{gl4} 
%{\bf Hard-disk architecture.} Guest lecture for ECE160 Multimedia Systems, 
%UCSB, Spring 2002.

%\bibitem{xtream-sony} 
%{\bf The XTREAM multimedia system.} Research seminar (with Sony Research), UCSB, \\ 
%February 2002.

%\bibitem{gl5} 
%{\bf Disk admission control.} Intern project presentation at 
%Sony 550 DMV, San Francisco, California, Summer 2001.

%\bibitem{scsibench-talk} 
%{\bf Disk feature extraction.} Research seminar (with Sony Research), UCSB, Winter 2001.

%\bibitem{hpl-tal} 
%{\bf OS memory error susceptibility.} Intern project presentation at 
%HP Labs, Palo Alto, California, Summer 2000.
\end{thebibliography}


%\section{Invited Lectures}
%\begin{thebibliography}{22.}

%\bibitem{proof} Computational complexity and proof-theoretic power in
%  typed lambda calculi.  \newblock \emph{American Mathematical Society
%    Summer Meeting,} 1982.  \newblock Abstract appeared in {\em
%    Abstracts of the American Mathematical Society,} 3(5):376, August
%  1982.  \newblock Results from paper~\cite{type}.
%\end{thebibliography}

%\section{Editorial Duties}
%\begin{enumerate}
%\end{enumerate}

%\section{Software Research Projects}

%\pagebreak


\section{Awards, Fellowships, Honors}
\begin{tabular}{ll}
2018 	& SAP Catalyst. \\
2015 	& IEEE Senior Member. \\
2015-present 	& Member of IEEE Computer Society Industry Advisory Board. \\
2010 	& Google OC Award for Google File System. \\
2007--2013      & Member of Google Hiring Committee for new-grad PhDs. \\
2003 	& President's Work Study Award for 2003-2004, UCSB. \\
2002	& Graduate Student Travel Grant, Graduate Division, UCSB.\\
1999 	& Dean's Fellowship, College of Engineering, UCSB. \\
1999 	& Teaching Assistantship and Tuition Fellowship for 1999-2003, 
		CS Department, UCSB. \\
%1997	& Travel Grant to Brazil from local industry, \v{S}abac, Serbia.\\
%1993-1998 & Fellowship from the Ministry of Education, Republic of Serbia.\\
1993 	& Second place on the classification exam in Physics, University of Belgrade. \\
1988-1992 & First place awards on Serbian state Physics competitions in 
		1988, 1990, and 1992.
\end{tabular}


\section{Refereeing}

{\bf Journals:}
{\em ACM Transactions on Storage, IEEE Transactions on Parallel and Distributed Systems},
{\em The Computer Journal} (published by Oxford University Press on behalf of
British Computer Society), {\em Software Practice and Experience}
(published by John Wiley \& Sons, Ltd.), IEEE Transactions on Computers, 
IEEE Transactions on Cloud Computing, and Applied Computing and Informatics.

{\bf Conferences:}
{\em IEEE Infocom 2005, International Conference on Database Systems for
Advanced Applications (DASFAA 2004), IEEE Conference on Multimedia and
Expo (ICME2004, ICME2005), IEEE MMM 2006, ACM Multimedia 2006, IEEE (MSST2010) Symposium on 
Massive Storage Systems and Technologies, ACM/IFIP/USENIX Middleware 2016, UCSB Computer Science Capstone 2019,
IEEE International Conference on Cloud Engineering (IC2E) 2019}.


%\section{Program Committees}
%\begin{enumerate}
%\end{enumerate}


%\section{Professional Societies}
%Member of the ACM, IEEE, IEEE Computer Society, and Usenix.


%\pagebreak


%\section{Professional and University Service}
%\vspace{-6pt}
%\begin{enumerate}
%
%\item Maintaining web-page on Operating Systems Storage for 
%IEEE DSO, 2002-present.
%
%\end{enumerate}

%\pagebreak

%\section{Community Service}
%\section{Research Grants and Contracts}

%%% software projects

%%% skills


%\section{Professional Meetings}
%\vspace{-6pt}

%\begin{enumerate}
%\item {\em Sixth Usenix Operating Systems Design and Implementation (OSDI)}, 
%San Francisco, California, December 2004.

%\vspace{-5pt}
%\item {\em Third Usenix File and Storage Technology (FAST)}, San Francisco, 
%California, March 2004.

%\vspace{-5pt}
%\item {\em First ACM International Workshop on Video Surveillance}, Berkeley, 
%California, November 7, 2003.

%\vspace{-5pt}
%\item {\em 11th ACM Multimedia}, Berkeley, California, November 2003.

%\vspace{-5pt}
%\item {\em Second Usenix File and Storage Technology (FAST)}, San Francisco, 
%California, March 2003.

%\vspace{-5pt}
%\item {\em Workshop on Robustness Analysis Tools with Applications to
%the Biological and Physical Sciences: The Challenge of Complexity}, Kavli Institute 
%for Theoretical Physics, University of California, Santa Barbara, March 2003.

%\vspace{-5pt}
%\item {\em 10th ACM Multimedia}, Juan Les Pins, France, December, 2002.

%\vspace{-5pt}
%\item {\em IEEE International Conference on Multimedia and Expo}, Lausanne, 
%Switzerland, August, 2002.

%\vspace{-5pt}
%\item {\em ACM SIGMOD/PODS}, Santa Barbara, California, May 2001.

%\vspace{-5pt}
%\item {\em Java(tm) Virtual Machine Research and Technology Symposium}, Monterey,
%California, April, 2001.

%\vspace{-5pt}
%\item {\em 18th IEEE Symposium on Mass Storage Systems and Technologies
%(in cooperation with the Ninth NASA Goddard Conference on Mass Storage
%Systems and Technologies)}, San Diego, California, April, 2001.

%\vspace{-5pt}
%\item {\em YU INFO'99 Symposium on Computer Sciences and Informational Technologies}, 
%Kopaonik, Serbia, Yugoslavia, March 1999.

%\vspace{-5pt}
%\item {\em Seminar on Telecommunications}, National Technical University of Athens, 
%Greece, March 1998.

%\vspace{-5pt}
%\item {\em Seminar on Physics}, Research Center Petnica, Serbia, Yugoslavia, 1991.
%\end{enumerate}

%\section{Personal}
%My hobbies include Linux and other open-source projects, Internet cultures, 
%cycling, mountain biking, skydiving, little karate, even less aikido,
%skiing, scuba diving, music, travelling... and C.

%\pagebreak 

%\section{References}
%\vspace{6pt}
%
%Available upon request.

%\begin{tabular}{ll}
%{\bf Dr. Edward Y. Chang}                         & {\bf Dr. Klaus E. Schauser}\\    
%Department of Electrical and Computer Engineering & Department of Computer Science\\
%University of California, Santa Barbara       & University of California, Santa Barbara\\ 
%UCSB, CA 93106					  & UCSB, CA 93106 \\
%Tel: (805) 893-2971                               & E-mail: {\em schauser@cs.ucsb.edu} \\
%E-mail: {\em echang@ece.ucsb.edu} & \\
%&\\
%{\bf Dr. Tao Yang}                        &  {\bf Dr. Elizabeth M. Belding-Royer}\\    
%Department of Computer Science            &  Department of Computer Science\\          
%University of California, Santa Barbara   &  University of California, Santa Barbara\\
%UCSB, CA 93106				  &  UCSB, CA 93106\\
%Tel: (805) 893-4384                       &  Tel: (805) 893-3411\\                     
%E-mail: {\em tyang@cs.ucsb.edu}           &  E-mail: {\em ebelding@cs.ucsb.edu}\\
%&\\
%{\bf Dr. Anurag Acharya}    	& \\
%Google, Inc.                 	& \\
%1600 Amphitheatre Pkwy. 	& \\
%Mountain View, CA 94043    	& \\
%Tel: (650) 330-0100           	& \\
%E-mail: {\em acha@google.com} 	& \\
%\end{tabular}

%\section{Consulting and Visiting}

%\section{Courses Taught}

\end{document}
